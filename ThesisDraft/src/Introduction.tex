\subsection{Summary}
Open source software (OSS) repositories contain a large amount of data that has been accumulated along the software development process. Not only source code but also metadata available from different related sources, e.g. communication channels, bug tracking systems, is beneficial to the development process once it is properly mined. Research has been performed to understand and predict software evolution, exploiting the rich metadata available at OSS repositories. This allows for the reduction of effort in knowledge acquisition and quality gain. Developers can leverage the underlying knowledge if they are equipped with suitable tools. For instance, it is possible to empower IDEs by means of tools that continuously monitor the developer's activities and contexts in order to activate dedicated recommendation engines \cite{Ponzanelli:2014:MST:2597073.2597077}. 

%To aim for software quality, for developers it is necessary to understand how similar, mature projects are developed.   it is necessary

To aim for software quality, developers normally build their project by learning from mature OSS projects having comparable functionalities. To this end, the ability to search for similar software projects with respect to different criteria such as functionalities and dependencies plays an important role in the development process. Two projects are deemed to be similar if they implement some features being described by the same abstraction, even though they may contain various functionalities for different domains \cite{McMillan:2012:DSS:2337223.2337267}. Understanding the similarities between open source software projects allows for reusing of source code and prototyping, or choosing alternative implementations \cite{Schafer:2007:CFR:1768197.1768208},\cite{10.1109/SANER.2017.7884605}, thereby improving software quality. Meanwhile measuring the similarities between developers and software projects is a critical phase for most types of recommender systems \cite{DBLP:conf/rweb/NoiaO15},\cite{Sarwar:2001:ICF:371920.372071}. Similarities are used as a base by both content-based and collaborative-filtering recommender systems to choose the most suitable and meaningful items for a given item \cite{Schafer:2007:CFR:1768197.1768208}. Failing to compute precise similarities means concurrently adding a decline in the overall performance of these systems. Nevertheless, measuring similarities between software systems has been considered as a daunting task \cite{Chen:2015:SFD:2684822.2685305},\cite{McMillan:2012:DSS:2337223.2337267}. Furthermore, considering the miscellaneousness of artifacts in open source software repositories, similarity computation becomes more complicated as many artifacts and several cross relationships prevail.

In recent years, considerable effort has been made to provide automated assistance to developers in navigating large information spaces and giving recommendations. Though remarkable progress can be seen in this field, there is still room for improvement. To the best of our knowledge, most of the existing approaches consider the constituent components of the OSS ecosystem separately, without paying much attention to their mutual connections. There is a lack of a proper scheme that facilitates a unified consideration of various OSS artifacts and recommendations. 

CROSSMINER\footnote{\url{https://www.crossminer.org}} is a research project funded by the EU Horizon 2020 Research and Innovation Programme, aiming at supporting the development of complex software systems by \textit{i)} enabling monitoring, in-depth analysis and evidence-based selection of open source components, and \textit{ii)} facilitating knowledge extraction from large OSS repositories \cite{10.1007/978-3-319-74730-9_33}. In the context of the project, we work towards an advanced Eclipse-based IDE providing intelligent recommendations that go far beyond the current \emph{code completion-oriented} practice. Among others, an indispensable functionality is to find a set of similar OSS projects to a given project with respect to different criteria, such as external dependencies, application domain, or API usage \cite{NDRDSEAA2018},\cite{DBLP:conf/iir/NDD013}.

The purpose of this thesis is the implementation of two approaches, MUDABlue and Clan, with the aim of compare the results of new tool by CROSSMINER development team (CrossSim). CROSSSIM (Cross Project Relationships for Computing Open Source Software Similarity), is an approach that makes use of graphs for rep-resenting different kinds of relationships in the OSS ecosystem. In particular, with the adoption of the graph representation, we are able to transform the relationships among non-human artifacts, e.g. API utilizations, source code, interactions, and humans, e.g. developers into a mathematically computable format, i.e. one that
facilitates various types of computation techniques. Naturally this kind of approaches has to be evaluated, and confronted with others similar tools. My work helps addressing this challenge providing these two tools and evaluating all the results to show how nice is CrossSim.